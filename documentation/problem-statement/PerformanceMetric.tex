\section{Performance Metric}
\label{sec:PerformanceMetric}
% The criteria that we intend to hold ourselves to are:
For this project to be considered complete and have a successful project, there should be the various criterion that needs to be met. The app that we will end up creating has a few requirements so far, although this list will grow as our research develops. The first is that it uses the new WebVR API and utilizes enough of the API’s functionalities that someone else could use the app as a reference. Another is that it is a usable app with a friendly user interface and user experience, including the surrounding website. A third is that it is open source and well documented enough that others can follow it to create their own web augmented reality or virtual reality application based on this one. Although there are requirements for the website that will be our end product, there is not any specification about whether the virtual reality experience must be a video game or simulation, for example.

%hardware agnostic
Another requirement of the website is that it is hardware agnostic. That is to say that it can be run on any device with a modern browser and still provide a good user experience. One important aspect of this that the client pointed out is recognizing whether the device the content is being served to is mobile or a desktop. If it is a desktop, the site should serve the full resolution textures for the greatest user experience. However, if the client is a mobile device, the site should serve lower quality textures so that the phone can render it without too much strain.

Parallel to the hardware agnostic is the requirement that the frame budget is met. For VR, there is about 11 milliseconds of time for each frame to render. Should the content in the application exceed 11 milliseconds of load times, users will experience frame drops and motion sickness. The frame budget should never be exceeded. This includes maintaining a stable 90 frames per second in each viewing lens.

Although the end product is an important part of the deliverable, the research we did to come up with the project and all of the issues we encountered along the way are also very valuable because of the nature of the project. Therefore, we will need to be very thorough when documenting this research and these issues so that other people can come behind us and use what we have learned to create other great web AR and VR products. 

As of yet, there are no quantifiable performance metrics that need to be met for the project, but these will be updated as our research progresses and we begin to have a better understanding of what our final deliverable will be. 