\section{Specific requirements}
\subsection{External interfaces}
\begin{itemize}
    \item Users may interface with the website using:
    \begin{itemize}
    \item Mobile devices with positional tracking
        \item Head mounted displays, whether they are opaque, transparent, or utilize video pass-through
        \item Fixed displays with head tracking capabilities
    \end{itemize}
    \item Users will be able to interact with objects in the environment through their controller (or simulated/virtual controller)
    \begin{itemize}
        \item Translate objects
        \item Rotate objects
        \item Pick up/grasp objects
        \item Drop/throw objects
    \end{itemize}
    \item Users must be able to change physical constants.  Potential mechanisms for doing that:
    \begin{itemize}
        \item Physical Constants Dashboard with levers/switches/dials to change the environment
        \item Floating user interfaces around the objects with icons/bars and other traditional UI features
    \end{itemize}
    \item User can spawn in a multitude of objects stored on the website through a spawn menu
        
\end{itemize}

\subsection{Functions}
\begin{itemize}
    \item Utilize all core features of WebXR including:
    \begin{itemize}
        \item Headset location and orientation
        \item Controller location and orientation
        \item Multi-platform compatibility
        \item Parsing input from XR devices
    \end{itemize}
    \item The website will load the correct VR environment depending on the hardware that the user has.
    \item Website will pause when a controller has been disconnected and resume when the connection is reestablished.
    \item Website will notify user if web browser/hardware is not compatible with WebXR
    \item Render environment and physics in real time
\end{itemize}
\subsection{Usability requirements}
Offer a way of manipulating objects in the simulation for all supported platforms.  On a phone this might mean having a virtual manipulator at a fixed location from the user's face. On devices like the Oculus Rift, this would probably mean the hand controllers.

We want to pair a non-technical-user friendly physics demonstration with a technical-user friendly code base.

\subsection{Performance requirements}
\begin{itemize}
    \item Latency must not exceed 11 milliseconds within the simulation.
    \item Re-factor to distribute to multiple web workers if needed to achieve performance requirements.
    \item Meet an average 60 frames per second, appropriately degrading the content to meet that budget on mobile phones and other low powered devices.
    \item HMDs should be running at 90 fps
\end{itemize}
%\subsection{Logical database requirements}
%Content - See IEEE 9.5.14
%\subsection{Design constraints}
\subsection{Software system attributes}
\begin{itemize}
    \item Landing page will be compatible with all evergreen browsers
    \item Website will be compatible with browsers that support WebXR
    \item Website will be compatible with
    \begin{itemize}
        \item Magic Window compatible mobile devices using compatible browsers.
        \item Google Daydream
        \item HTC Vive
    \end{itemize}
    \item Website will run on multiple devices and operating systems (Windows, Android, Mac)
\end{itemize}