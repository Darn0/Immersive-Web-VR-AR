\documentclass[onecolumn, draftclsnofoot,10pt, compsoc]{IEEEtran}
\usepackage{graphicx}
\usepackage{url}
\usepackage{setspace}
\usepackage{abstract}
\usepackage{geometry}
\geometry{textheight=9.5in, textwidth=7in}
\parindent = 0.0 in
\parskip = 0.1 in

% 1. Fill in these details
\def \CapstoneTeamName{WebXR Team}
\def \CapstoneTeamNumber{47}
\def \GroupMemberOne{Brooks Mikkelsen}
\def \GroupMemberTwo{Evan Brass}
\def \GroupMemberThree{Jonathan Jones}
\def \GroupMemberFour{Brandon Mei}
\def \GroupMemberFive{Tim Forsyth}
\def \CapstoneProjectName{Creating Immersive Experiences on the Web using VR and AR.}
\def \CapstoneSponsorCompany{Intel}
\def \CapstoneSponsorPerson{Alexis Menard}

% 2. Uncomment the appropriate line below so that the document type works
\def \DocType{		Problem Statement
				%Requirements Document
				%Technology Review
				%Design Document
				%Progress Report
				}
			
\newcommand{\NameSigPair}[1]{\par
\makebox[2.75in][r]{#1} \hfil 	\makebox[3.25in]{\makebox[2.25in]{\hrulefill} \hfill		\makebox[.75in]{\hrulefill}}
\par\vspace{-12pt} \textit{\tiny\noindent
\makebox[2.75in]{} \hfil		\makebox[3.25in]{\makebox[2.25in][r]{Signature} \hfill	\makebox[.75in][r]{Date}}}}
% 3. If the document is not to be signed, uncomment the RENEWcommand below
%\renewcommand{\NameSigPair}[1]{#1}

%%%%%%%%%%%%%%%%%%%%%%%%%%%%%%%%%%%%%%%
\begin{document}
\begin{titlepage}
    \pagenumbering{gobble}
    \begin{singlespace}
    	%\includegraphics[height=4cm]{coe_v_spot1}
        \hfill 
        % 4. If you have a logo, use this includegraphics command to put it on the coversheet.
        %\includegraphics[height=4cm]{CompanyLogo}   
        \par\vspace{.2in}
        \centering
        \scshape{
            \huge CS Capstone \DocType \par
            {\large\today}\par
            \vspace{.5in}
            \textbf{\Huge\CapstoneProjectName}\par
            \vfill
            {\large Prepared for}\par
            \Huge \CapstoneSponsorCompany\par
            \vspace{5pt}
            {\Large\NameSigPair{\CapstoneSponsorPerson}\par}
            {\large Prepared by }\par
            Group \CapstoneTeamNumber\par
            % 5. comment out the line below this one if you do not wish to name your team
            %\CapstoneTeamName\par 
            \vspace{5pt}
            {\Large
                \NameSigPair{\GroupMemberOne}\par
                \NameSigPair{\GroupMemberTwo}\par
                \NameSigPair{\GroupMemberThree}\par
                \NameSigPair{\GroupMemberFour}\par
                \NameSigPair{\GroupMemberFive}\par
            }
            \vspace{20pt}
        }
        %\renewcommand{\abstracttextfont}{\sffamily}
        \begin{abstract}
        % 6. Fill in your abstract    
        	Web virtual reality and augmented reality are technologies that are just getting started and are very cutting edge. In the past few years, a standard has come out specifying an API for creating web apps that use virtual reality and augmented reality. Because this is such a new field, there are not many projects that provide examples to other developers trying to use this technology. Group 47's job is to research and develop an application that showcases the capabilities of the WebXR API. Some proposed projects include a physics simulation, data visualization, event seating, and simple video games. Each project is aimed at providing an interactive VR experience that does not require client-side software. The project will be considered complete when it has a working prototype, satisfactory UX, maintained compatibility, is interactive and remains within the VR frame budget of 11 milliseconds.
        \end{abstract}     
    \end{singlespace}
\end{titlepage}
\newpage
\pagenumbering{arabic}
\tableofcontents
% 7. uncomment this (if applicable). Consider adding a page break.
%\listoffigures
%\listoftables
\clearpage

% 8. now you write!
\section{Problem Statement}
\label{sec:problemStatement}
% Intro: What is the context of our project?
Virtual reality (VR) and augmented reality (AR) are both still in the early stages of production. Because the industry is so new, entrepreneurs and industry leaders are still coming up with new ways virtual reality and augmented reality can be harnessed to create great products.
% 1. Ways that VR/XR can be beneficial:

% - Problem: VR Experiences require large downloads
Currently, most products require being downloaded onto a computer before being run. Examples of this method of distribution include steam games that support virtual reality and product training simulations that are downloaded to a computer. While this is great for larger apps that require gigabytes of textures and 3D object files downloaded before they can even run, many smaller AR or VR apps might not need as much overhead. 
% - Solution: Websites are always up to date and don't require installation

This is where Web virtual reality and augmented reality comes into play. New web specifications have been created that call for a virtual reality web API (application programmer interface, the tools developers use to build applications on top of an existing framework) called the WebXR API. The idea is that a user could navigate to a website via a typical web browser such as Google Chrome, Safari, or Firefox and without downloading client side software, render VR or AR to a headset such as the Oculus Rift or Microsoft HoloLens. This would allow users to quickly view new AR and VR experiences in the same way that streaming makes watching videos online much faster and easier.
% - Problem: The existing api (WebVR) couldn't handle handheld controllers or mixing Computer Generated Content with the real world
% - Solution: Web XR exposes endpoints for controller location data and reserves space to implement augmented reality where computer graphics mixes with camera data or the real world (Multiple Layers in the WebXR standard)

The WebXR API is the successor to an API called WebVR. WebVR provides simple virtual reality functionality, but doesn't have built in support for augmented reality. WebVR is compatible with lots of different VR devices and browsers but has struggled in its absence of VR controller support. Users have been unable to interact with the virtual environment they have been presented with. This has been a barrier to most developers who want to create immersive VR experiences for their users. Developers end up needing to add the same workarounds to each project, so the community decided that a new API specification is needed. 

% - Problem: VR/AR experiences rarely have a fallback for those with vision or auditory disabilities.

% - Solution: The web has native support for screen readers and announcing content changes

% - Problem: VR/AR experiences often require large downloads and target either low end phones or high end headsets
% - Solution: Websites have a long history / Solid design patterns for "progressively enhancing" a website from mobile friendly to desktop friendly

% 2. WebXR has tremendous advantages, Why isn't it being used?

% - Problem: Lack of examples / information on how much effort it will require
% - Problem: Lack of stability (old WebVR will require poly-filling)
% - Problem: The news needs to be spread that the Kronos group is standardizing OpenXR and that the web will be getting it in parallel
Currently, because the WebXR API is in such early stages (it isn’t even available for some browsers yet), there are not a lot of examples for developers to use as a reference point while creating new experiences. Without a strong set of open source examples and articles where developers can learn from others’ mistakes, creating WebXR projects becomes a lot more time consuming. There is also a lot more troubleshooting and research involved in the process before developers can create a new experience.

We, Group 47, have the honor of being pioneers in this field as there are little to no applications that showcase the abilities of WebXR. Our central purpose is to essentially develop a leading example for WebXR applications. It is imperative that we make an enjoyable VR experience that enhances the web user experience.
\section{Solution}
\label{sec:Solution}
% 1. What we're doing:
Our group will work to develop an application that is able to demonstrate the WebXR abilities. Specifically, we want to make use of VR controllers in our web application. Since WebVR is already well known and doesn’t make use of controllers, our job is to create an experience that you are able to interact with using WebXR.

% 2. How we're going to do it:
The goal is to include many common features that potential developers may use in trying to develop their own WebXR experiences so that, in the future, our work can be used as a reference. This website will eventually be hosted on Intel’s open source portal, 01.org. From there, people in the open source community can try out the technology and see the source code if they are interested in creating a web virtual reality or augmented reality project of their own. 

% 3. Why is VR the right choice for our project's content?
Research is an integral part in ensuring that the development process is smooth. As our client noted, we will need to research the best way create the VR experience before we begin development. Research will also need to be done on the benefits brought by adding AR or VR functionality to a website.

% 4. What will it look like:
% - What makes each project something that we'll be able to showcase WebXR through?
% Physics - Lots of controller interactions, example of progressive enhancement (basic physics equations -> physics playground)
Many ideas have been thrown around already. These include a physics simulation where users can explore the laws of physics on objects in the scene such as spheres. Users would be able to scale objects, throw them around, create tracks for testing physics and change the global physics values in the scene.

% Event Seating - Real life use case that could affect sales, Not as much controller interaction
Another proposed application was an event seating viewer. Users, while purchasing seating tickets for a sporting event or live show are often unsure about the view that their seats have. They could place themselves into a virtual reality environment where they could view a 3D model of the stadium or theatre from their chosen seat before purchase.

% Data Visualization - Lots of controller use, progressive enhancement (2d graph -> 3d graph with interactability)
Also discussed were Data Visualization applications. Some scatter plots and graphs are incredibly difficult to read when they incorporate gigantic amounts of data into a small 2D space. These graphs are often hard to visualize. VR makes data visualization easier because you are able to inspect the data in three dimensions. The additional dimension removes the clutter and restrictions of a 2D space, enhancing the user experience. Interaction with the scene can come in the form of scaling and rearranging the data.

% Battleship/Tower Defense - Good controller use, progressive enhancement (simple battleship -> awesome battleship)
Several games were proposed such as Battleship and Tower Defense. Everyone likes a good video game. So naturally a good solution for this problem is to provide some sort of VR gaming experience on the web.

Most of the proposed solutions were defined by how they are able to deliver an experience that the user cannot normally get without many additional steps. Most important is how each project is able to deliver some impact to those who experience them. The goal of this project is to deliver an experience that garners interest in the WebXR platform and its capabilities.

As we do this research, we will work with our client to develop the actual app we are creating as an example to the community. This enables more content creation because the web has great scaling and is easily accessible to millions of internet user instantly. In this respect, we will get some degree of freedom in choosing the deliverable project. 

\section{Performance Metric}
\label{sec:PerformanceMetric}
% The criteria that we intend to hold ourselves to are:
For this project to be considered complete and have a successful project, there should be the various criterion that needs to be met. The app that we will end up creating has a few requirements so far, although this list will grow as our research develops. The first is that it uses the new WebVR API and utilizes enough of the API’s functionalities that someone else could use the app as a reference. Another is that it is a usable app with a friendly user interface and user experience, including the surrounding website. A third is that it is open source and well documented enough that others can follow it to create their own web augmented reality or virtual reality application based on this one. Although there are requirements for the website that will be our end product, there is not any specification about whether the virtual reality experience must be a video game or simulation, for example.

%hardware agnostic
Another requirement of the website is that it is hardware agnostic. That is to say that it can be run on any device with a modern browser and still provide a good user experience. One important aspect of this that the client pointed out is recognizing whether the device the content is being served to is mobile or a desktop. If it is a desktop, the site should serve the full resolution textures for the greatest user experience. However, if the client is a mobile device, the site should serve lower quality textures so that the phone can render it without too much strain.

Parallel to the hardware agnostic is the requirement that the frame budget is met. For VR, there is about 11 milliseconds of time for each frame to render. Should the content in the application exceed 11 milliseconds of load times, users will experience frame drops and motion sickness. The frame budget should never be exceeded. This includes maintaining a stable 90 frames per second in each viewing lens.

Although the end product is an important part of the deliverable, the research we did to come up with the project and all of the issues we encountered along the way are also very valuable because of the nature of the project. Therefore, we will need to be very thorough when documenting this research and these issues so that other people can come behind us and use what we have learned to create other great web AR and VR products. 

As of yet, there are no quantifiable performance metrics that need to be met for the project, but these will be updated as our research progresses and we begin to have a better understanding of what our final deliverable will be. 

\end{document}