\section{Introduction}
Building virtual reality applications is hard and so is building a physics simulation. Developing both at the same time is even more difficult. There are lots of moving parts to a project such as this. For the visual/audio experience to materialize, there needs to be graphical, physics engine, and audio components. Graphical APIs get images on the webpage, physics APIs create realistic image movement and audio APIs immerse the user with sound. Implementing these features on their own is difficult enough. Thankfully, web developers have been hard at work creating a multitude of libraries that provide these services. Choosing the right libraries for the job, however, can be difficult. These libraries need to work together and most importantly, with WebXR. Under the assumption that WebXR developers have been building upon their past API, WebVR, the ideal libraries will be those that have built in support for WebVR. There must also be sufficient documentation associated with each library.
