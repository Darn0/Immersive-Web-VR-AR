\section{Audio API}
The goal of VR is immersion and audio is an incredibly important part of this process. For this project to be immersive, it needs event driven spatial audio and an audio API that can provide it. Spatial audio is a full-sphere surround sound technique that utilizes the three dimensions to mimic real life audio \cite{r12}. This is the only way to correctly convey sound to the user in a virtual reality simulation. Much like the other components in this review, this part of the project does not need to be created from scratch as other developers have done that job already.

\subsection{WebAudio}
The WebAudio API provides a powerful and versatile system for controlling audio on the Web that gives developers the ability to choose audio sources, add effects to audio, create audio visualizations, apply spatial effects and much more \cite{r13}. WebAudio works by using a series of audio nodes in an audio routing graph to bring organized audio framework to a webpage \cite{r13}. It is just about the only audio API that is used world wide on the internet. Part of the reason for this is the fact that it was developed by the W3C (World Wide Web Consortium). As far as VR development goes, Mozilla has said that “It's really useful for WebXR and gaming. In 3D spaces, it's the only way to achieve realistic audio.” \cite{r14} There is an enormous amount of documentation associated with WebAudio, mostly from the Mozilla developer site.

By itself, web audio is already easy to use and implement into web applications. Still, there are many libraries that simplify its use even further and add features themselves. The following APIs wrap around basic WebAudio and abstract it.

\subsection{HowlerJS}
HowlerJS makes working with audio and WebAudio in Javascript easy. It is a single API for all audio needs that defaults to WebAudio and falls back to HTML5 if needed \cite{r15}. It is well optimized, easy to use, modular and lightweight. Additionally, it is compatible with most modern browsers because of its fallback to HTML5 \cite{r15}. HowlerJS has stereo panning and 3D spatial audio support which is very important to this project. Finally, it has full codec support, meaning that all audio file types are compatible \cite{r15}. As a WebAudio API, it is one of the most popular in its field and has developed a large community. As a result, this API has extensive documentation.

\subsection{Waud}
Waud is essentially the same as HowlerJS. It has all the same features and support for spatial audio. Like Howler, it abstracts WebAudio, defaults to it, and falls back on HTML5 on unsupported browsers \cite{r16}. Both have zero dependencies and both are fully modular. One key difference however is the absence of volume fading in Waud. It is important to keep as many options open as possible when dealing with audio so this is a clear disadvantage to HowlerJS.
