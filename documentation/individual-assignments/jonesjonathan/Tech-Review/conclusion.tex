\section{Conclusion}
For this project to be successful, the right technologies must be chosen for the right components. For the graphical, physical engine and auditory components there are many options. Each with its own advantages and disadvantages. The sheer amount of Javascript libraries that area available for 3D visualizations is staggering and can make this choice unclear. For this project, however, there are clear choices for each component.

The physics engine should be Cannon.js. While Oimo.js and Ammo.js are both great physics engines, only one Cannon.js built directly into the A-Frame API. As it seems that this project will be making use of A-Frame, it only makes sense to use what works with it. This along with the extensive documentation that Cannon.js has makes this an easy choice.

The graphics API must be Three.js. This API stands out as a definitive choice among the three APIs listed. Like Cannon.js, Three.js is built into A-Frame which makes it highly desirable. Its incredibly feature richness and vibrant community is another factor in this recommendation. It is important that there is lots of support behind each component. Success is made all too easy when the entire world of web developers is at the back of the API being used by a project.

For the audio API it was another easy choice. HowlerJS is the favorite among web developers for 3D audio and spatialization. Its wide compatibility and abstraction of WebAudio make it simple and easy to use for beginners in the audio arena. Like the others, HowlerJS has the most extensive documentation among its competitors.

Each technology has its advantages and disadvantages but these technologies will only serve to improve the development experience of this project.

